% PROGETTO BASI DI DATI 2015 , MERLINO GIUSEPPE is CEO and VIOLETTO MIKI is SGUATTERO

\documentclass[a4paper,twoside]{article}

\usepackage[margin=3cm]{geometry}
\usepackage[utf8]{inputenc}
\usepackage[main=italian, english]{babel}
\usepackage{graphicx}
\usepackage{rotating}
\usepackage{verbatim}
\usepackage{listings}
\usepackage{hyperref}
\hypersetup{colorlinks=true,linkcolor=black,urlcolor=blue}

\author{Miki Violetto (per ora)}
\title{Relazione progetto BiciRent}
\date{\today}

\begin{document}
\maketitle

\newpage
\tableofcontents

\newpage
\section{Link utili}
\begin{itemize}
\item \textbf{http://www.goodbikepadova.it/Default.aspx} : GoodBike, dove prendiamo spunto per il progetto
\item \textbf{www.lucidchart.com} : programma per fare i grafici dei schemi concettuale e logico
\end{itemize}

\section{Abstract}
BiciRent è un servizio di bike sharing di biciclette ispirato a Goodbike presente qui a Padova.\newline
Permette grazie ad una tessera personale, rilasciata a chiunque ma con sconti per studenti e turisti, di noleggiare in qualsiasi momento una qualsiasi delle bici presenti su delle apposite aree grazie ad una semplice tessera e utilizzarla per muoversi in centro città, lasciandola poi in un'altra area dove sia presente un posto libero.\newline
La tessera può avere validità giornaliera o per un singolo fine settimana (sabato-domenica), oppure essere collegata ad un abbonamento di varie taglie: settimanale, mensile, trimestrale, semestrale o  annuale.\newline
Sono presenti, oltre alle normali bici che hanno integrato un posto aggiuntivo per bambini, biciclette con pedalata assistita e biciclette con 2 seggiolini;\newline
Le aree apposite sono prsenti in molti punti della città e munite si sicure automatiche antifurto e sono sempre presenti sia spazio per lasciare la bici noleggiata sia biciclette da noleggiare, grazie a varie convenzioni con ditte esperte di logistica che si occupano a muovere le bici tra le varie aree in base alle varie disponibilità e di portare le biciclette segnalate rotte ai meccnici con cui siamo convenzionati.\newline
Il servizio tende ad essere automatizzato, e fornisce all'abbonato di controllare facilmente se in una data area ci siano presenti biciclette, ci siano posti vuoti e può direttamente dall'area o dal sito avvertire se ci sono presenti biciclette con rotture.\newline
Nell'ultimo mese il servizio ha registrato più di 150 nuovi abbonamenti, con un utilizzo medio di circa 300 spostamenti e una media di 100 richieste di intervento (tra mancanza di bici o spazio e rotture);\newline
inoltre è stata richiesta la composizione delle bici nelle aree (presenza bici e posti e quali bici) almeno 800 volte ogni giorno, circa ma con punte di quasi 100 volte nel giro di una sola mezz'ora (ad esempio nelle ore di punta).

\section{Descrizione testuale dei requisiti e operazioni tipiche}

\section{Progettazione concettuale}

\section{Progettazione logica}

\section{Implementazione dello schema logico}

\section{Query, procedure, trigger e funzioni}

\section{Interfaccia web}

\end{document}