% PROGETTO BASI DI DATI 2015 , MERLINO GIUSEPPE is CEO and VIOLETTO MIKI is SGUATTERO

\documentclass[a4paper,twoside]{article}

\usepackage[margin=3cm]{geometry}
\usepackage[utf8]{inputenc}
\usepackage[main=italian, english]{babel}
\usepackage{graphicx}
\usepackage{rotating}
\usepackage{verbatim}
\usepackage{listings}
\usepackage[colorlinks=true,linkcolor=black,urlcolor=blue]{hyperref} % ancore delle varie sezioni sul menu


\author{Giuseppe Merlino,Miki Violetto}
\title{Relazione progetto BiciRent}
\date{\today}

\begin{document}
\maketitle

\newpage
\tableofcontents
\newpage

%  link utili
%  http://www.goodbikepadova.it/Default.aspx : GoodBike, dove prendiamo spunto per il progetto
%  www.lucidchart.com : programma per fare i grafici dei schemi concettuale e logico

\section{Abstract}
BiciRent è un servizio di bike sharing ispirato a Goodbike presente qui a Padova.\newline
Permette grazie ad una tessera personale, rilasciata a chiunque ma con sconti per studenti e turisti, di noleggiare in qualsiasi momento una delle bici presenti in apposite stazioni collocate in diverse aree della città ed utilizzarla per muoversi, lasciandola poi in un'altra stazione dove sia presente un posto libero.\newline
La tessera può avere validità giornaliera, per un singolo fine settimana (sabato e domenica), oppure essere collegata ad un abbonamento di varie taglie: settimanale, mensile, trimestrale, semestrale o  annuale.\newline
Sono presenti, oltre alle normali bici che hanno integrato un posto aggiuntivo per bambini, biciclette con pedalata assistita o aventi 2 seggiolini;\newline
Le stazioni sono presenti in molti punti della città e sono munite di sicure automatiche antifurto. Sono sempre presenti biciclette da noleggiare e posti per lasciare la bici noleggiata , grazie a trasportatori, le biciclette rotte verranno aggiustate da meccanici convenzionati il giorno stesso in cui viene segnalata un'anomalia dai clienti.\newline
Il servizio tende ad essere automatizzato, permettendo all'abbonato di controllare facilmente se in una data stazione siano presenti biciclette o ci siano posti vuoti per depositare la bici noleggiata. Può segnalare se ci sono presenti biciclette con rotture.\newline
Nell'ultimo mese il servizio ha registrato più di 70 nuovi abbonamenti, con un utilizzo medio di circa 1700 noleggi giornalieri e una media di 200 richieste d'intervento al giorno (tra mancanza di bici, spazio per il deposito e rotture);\newline
inoltre viene richiesta la composizione dei posti di ogni singola stazione (presenza biciclette, tipi disponibili e posti liberi) almeno 3000 volte ogni giorno, ma con punte di quasi 900 richieste in una singola mezz'ora (ad esempio nelle ore di punta).

\section{Descrizione testuale dei requisiti e operazioni tipiche}
%Alias analisi dei requisiti XD
Si vuole realizzare una base di dati e una semplice applicazione web che contenga e gestisca le informazioni relative al
bike sharing, simulando inoltre le procedure automatiche realizzate dalle aree di deposito delle biciclette.

%cose dubbie-extra :
%meccanici-trasportatori esterni o interni (~2 tabelle e alcuni atributi)
%pezzi in custodia dei meccanici (tantissime cose in più, ~un altro database nel database)
%possibilità di query extra per le valutazioni e previsioni (tante query in più, non troppo lavoro comunque)

\subsection{Requisiti}
Sono necessarie le informazioni personali degli utenti per poterli eventualmente rintracciare :
\begin{itemize}
 \item nome e cognome
 \item data e luogo di nascita
 \item residenza (via, numero, città, cap)
 \item email-numero telefonico
 \item se studenti è necessario il codice di matricola in caso di universitari o id della card Iostudio, rilasciata dal Ministero dell'Istruzione agli studenti delle scuole superiori. 
\end{itemize}
Ogni tessera identifica un solo utente e ha un singolo piano:
\begin{itemize}%la tessera con questi attributi non va bene 
 \item codice univoco
 \item validità
 \item piano: giornaliero, weekend, settimanale, mensile, trimestrale,semestrale o annuale
\end{itemize}
Una tessera deve per forza avere un piano abbinato.\newline
Agevolazioni:
\begin{itemize}
 \item turisti, per i soli piani giornaliero o weekend (-15\% grazie ad un contributo della regione)
 \item studenti per gli abbonamenti da settimanale in su (-10\%).
\end{itemize}
Biciclette:
\begin{itemize}
 \item codice identificativo
 \item tipo: normali, pedalata assistita o con 2 seggiolini
 \item stato: in uso, a magazzino, danneggiate o in riparazione %e quelle disponibili?
\end{itemize}
Stazioni:
\begin{itemize}
 \item codice
 \item posizione (luogo, via e numero)
 \item postazioni presenti (colonnine)
 \item stato attuale della colonnina (occupata, libera, tipologia di bici presente) %quindi la colonnina potrebbe essere un'altra entità
%extra \item possibile collegamento preferenziato con meccanici o trasportatori, ad esempio un meccanico vicino ad una stazione
\end{itemize}
Meccanici e trasportatori:
\begin{itemize}
 \item nome e cognome
 \item residenza (via, numero)
 \item contatto telefonico e/o email
 \item per i soli meccanici quando hanno riparato una specifica bicicletta, e una descrizione del danno presente %italiano da rivedere
 \item per i soli trasportatori quando hanno trasportato una singola bicicletta, il luogo di partenza e quello di arrivo (stazione-colonnina)
\end{itemize}
Manutenzione delle stazioni:
\begin{itemize}
 \item il codice della colonnina riparata
 \item la data e l'orario della riparazione
 \item una descrizione del danno presente
\end{itemize}

\subsection{Operazioni tipiche}
Operazioni comuni effettuate nella quasi totalità dei casi:
\begin{itemize}
 \item Noleggio di una bicicletta : interessa da quale stazione e colonnina è stata prelevata la bici, l'orario di prelievo, il codice del mezzo e l'identificatore del tesserato
 \item Restituzione di una bicicletta : interessa su quale stazione e colonnina viene depositata, l'orario di deposito, il codice del mezzo e del cliente
 \item Controllo della disponibilità di biciclette o spazi liberi : interessa in quale stazione, quanti posti vuoti e quante biciclette ci sono nella stazione, in particolare che tipo di bici sono presenti
 \item Segnalazione di mancanza di bici(area vuota) o di una specifica categoria di bici, o di posti liberi(area piena) : interessa la stazione segnalata e il tipo di segnalazione; a questa segnalazione seguirà una richiesta di intervento in automatico
 \item Segnalazione di bicicletta rotta o di colonnina rotta : interessa il cliente che l'ha fatta, la stazione e colonnina da cui è stata inviata e l'orario; a questa seguirà una richiesta di intervento di riparazione in automatico
 \item Movimento di una bicicletta : interessa chi ha trasportato la bicicletta, l'orario e il numero identificativo del viaggio, il motivo del viaggio (per area piena o vuota, per riparazione della bicicletta), da dove proveniva la bici, dove è stata depositata e il codice identificativo della bici
 \item Riparazione di una bicicletta : interessa il meccanico che esegue la riparazione, il codice identificativo della bicicletta, l'orario di riparazione (inizio e fine, infatti il meccanico ha fino a 2 giorni di tempo per ripararla) e una descrizione del danno presente nella bicicletta
 \item Riparazione di una colonnina : interessa quale colonnina, l'orario e una descrizione del danno.
\end{itemize}
Operazioni meno comuni:
\begin{itemize}
 \item Ogni mese andrà richiesto un resoconto dei lavori di manutenzione sulle colonnine delle aree-stazioni e sulle biciclette
 \item Inoltre verrà richiesta una lista conteggiante tutte le bici trasportate da ogni trasportatore, necessaria per calcolare il pagamento mensile dovuto
 \item In casi eccezionali si può richiedere tutti gli utenti che hanno utilizzato una bici, quando e per quanti, e in che aree è stata la bici
 \item Può venire anche richiesto una lista dei mezzi danneggiati e in quali aree, e una statistica nel tempo di questi dati
%extra \item Necessarie per la valutazione e la previsione futura saranno necessarie varie statistiche richieste mensilmente, come una statistica sui trasporti delle bici, sull'utilizzo delle aree, ecc
\end{itemize}

\section{Progettazione concettuale}
%top-down , bottom-up, inside-out , misto

% credo di stare usando un inside-out modellando a mano a mano quello che leggo

\subsection{Prima modellazione concettuale}
Entità : \textbf{Utente}\newline
Attributi di \textbf{Utente} :
\begin{itemize}
 \item nome
 \item cognome
 \item data di nascita
 \item luogo di nascita
 \item indirizzo di residenza(via e numero)
 \item città di residenza
 \item cap
 \item email
 \item numero telefonico
\end{itemize}
Generalizzazione di \textbf{Utente} : \textbf{Studente}\newline
Attributo di \textbf{Studente} :
\begin{itemize}
 \item matricola
\end{itemize}
Entità : \textbf{Tessera}\newline
Attributi di \textbf{Tessera} :
\begin{itemize}
 \item codice (chiave)
 \item validità
\end{itemize}
Relazione : \textbf{Tesseramento} tra \textbf{Tessera} e \textbf{Utente} :
\begin{itemize}
 \item da \textbf{Tessera} a \textbf{Utente} 1 : 1
 \item da \textbf{Utente} a \textbf{Tessera} 1 : 1
\end{itemize}
Entità : \textbf{Piano}\newline
Attributi di \textbf{Piano} :
\begin{itemize}
 \item tipologia : giornaliero, weekend, settimanale, mensile, trimestrale, semestrale o annuale
\end{itemize}
Generalizzazione di \textbf{Utente} : \textbf{Turista}\newline
Relazione : \textbf{Abbonamento} tra \textbf{Tessera} e \textbf{Piano}
\begin{itemize}
 \item da \textbf{Tessera} a \textbf{Piano} 1 : 1
 \item da \textbf{Piano} a \textbf{Tessera} 1 : 1
\end{itemize}
Entità : \textbf{Bicicletta}\newline
Attributi di \textbf{Bicicletta} :
\begin{itemize}
 \item codice identificativo (chiave)
 \item tipo : normali, pedalata assistita o con 2 seggiolini
 \item stato : in uso, in magazzino, danneggiate, in riparazione
\end{itemize}
Entità : \textbf{Stazione}\newline
Attributi di \textbf{Stazione} :
\begin{itemize}
 \item codice (chiave)
 \item indirizzo (via e numero, luogo)
 \item postazioni presenti (numero di colonnine)
 \item stato attuale delle colonnine (occupate, libere, tipologia di bici presenti)
 \end{itemize}
Relazione : \textbf{Appartenenza} tra \textbf{Stazione} e \textbf{Colonnina}
\begin{itemize}
 \item da \textbf{Stazione} a \textbf{Colonnina} 1 : N
 \item da \textbf{Colonnina} a \textbf{Stazione} 1 : 1
\end{itemize}
Mi accorgo che il concetto 'stato attuale delle colonnine' non riesco a rappresentarlo, quindi decido di sviluppare questo attributo in una nuova Entità e Relazione.\newline
Entità : \textbf{Colonnina}\newline
Attributi di \textbf{Colonnina} :
\begin{itemize}
 \item codice numerico
\end{itemize}
Generalizzazione di \textbf{Colonnina} : \textbf{Colonnina occupata}\newline
Attributi di \textbf{Colonnina occupata} :
\begin{itemize}
 \item tipologia bicicletta
\end{itemize}
Relazione : \textbf{Posizionamento} tra \textbf{Colonnina occupata} e \textbf{Bicicletta}
\begin{itemize}
 \item da \textbf{Colonnina occupata} a \textbf{Bicicletta} 1 : 1
 \item da \textbf{Bicicletta} a \textbf{Colonnina occupata} 1 : 1
\end{itemize}
%
%%%da finire
%
Legami non rappresentati :\newline
Alcuni utenti possono avere agevolazioni in base alla loro tipologia e piano che scelgono sulla loro tessera :
\begin{itemize}
 \item  i turisti per i soli piani giornaliero o weekend (-15\% grazie ad un contributo della regione)
 \item i studenti per gli abbonamenti da settimanale in su (-10\%).
\end{itemize}

\section{Progettazione logica}

\section{Implementazione dello schema logico}

\section{Query, procedure, trigger e funzioni}

\section{Interfaccia web}

\end{document}