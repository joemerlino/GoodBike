% PROGETTO BASI DI DATI 2015 , MERLINO GIUSEPPE is CEO and VIOLETTO MIKI is SGUATTERO
%
% stato della relazione :
%
% abstract da rileggere e forse modificare
% analisi dei requisiti da cambiare un bel pò, aspetterei di farlo dopo aver fatto la progettazione concettuale, almeno l'inizio
% progettazione concettuale rimodellata, ci sarà ca sostituire gli itemize con strutture migliori
%
\documentclass[a4paper,twoside]{article}

\usepackage[margin=3cm]{geometry}
\usepackage[utf8]{inputenc}
\usepackage[main=italian, english]{babel}
\usepackage{graphicx}
\usepackage{rotating}
\usepackage{verbatim}
\usepackage{listings}
\usepackage[colorlinks=true,linkcolor=black,urlcolor=blue]{hyperref} % ancore delle varie sezioni sul menu


\author{Giuseppe Merlino,Miki Violetto}
\title{Relazione progetto BiciRent}
\date{\today}

\begin{document}
\maketitle

\newpage
\tableofcontents
\newpage

%  link utili
%  http://www.goodbikepadova.it/Default.aspx : GoodBike, dove prendiamo spunto per il progetto
%  www.lucidchart.com : programma per fare i grafici dei schemi concettuale e logico

\section{Abstract}
BiciRent è un servizio di bike sharing ispirato a Goodbike presente qui a Padova.\newline
Permette grazie ad una tessera personale, rilasciata a chiunque ma con sconti per studenti e turisti, di noleggiare in qualsiasi momento una delle bici presenti in apposite stazioni collocate in diverse aree della città ed utilizzarla per muoversi, lasciandola poi in un'altra stazione dove sia presente un posto libero.\newline
La tessera può avere validità giornaliera, per un singolo fine settimana (sabato e domenica), oppure essere collegata ad un abbonamento di varie taglie: settimanale, mensile, trimestrale, semestrale o  annuale.\newline
Sono presenti, oltre alle normali bici che hanno integrato un posto aggiuntivo per bambini, biciclette con pedalata assistita o aventi 2 seggiolini;\newline
Le stazioni sono presenti in molti aree della città e sono munite di sicure automatiche antifurto. Saranno sempre presenti biciclette da noleggiare e posti per lasciare la bici noleggiata grazie ad un servizio di trasporto e le biciclette rotte verrano sostituite da biciclette nuove.\newline
Il servizio tende ad essere automatizzato e permette all'abbonato di controllare facilmente se in una data stazione siano presenti biciclette o ci siano posti vuoti per depositare la bici noleggiata. Può segnalare se ci sono presenti biciclette con rotture.\newline
Nell'ultimo mese il servizio ha registrato più di 70 nuovi abbonamenti, con un utilizzo medio di circa 1700 noleggi giornalieri e una media di 200 richieste d'intervento al giorno (tra mancanza di bici, spazio per il deposito e rotture);\newline
inoltre viene richiesta la composizione dei posti di ogni singola stazione (presenza biciclette, tipi disponibili e posti liberi) almeno 3000 volte ogni giorno, ma con punte di quasi 900 richieste in una singola mezz'ora (ad esempio nelle ore di punta).

\section{Descrizione testuale dei requisiti e operazioni tipiche}
%Alias analisi dei requisiti XD
Si vuole realizzare una base di dati e una semplice applicazione web che contenga e gestisca le informazioni relative al
bike sharing, simulando inoltre le operazioni che è possibile eseguire tramite la colonnina.

\subsection{Descrizione testuale}
Sono necessarie le informazioni personali degli utenti per permettergli di registrarsi e comperare una tessera; saranno quindi obbligatori nome e cognome, la data e il luogo di nascita, le informazioni sulla residenza (via,numero civico,città e cap) e un'email, di aggiuntivo l'utente può inserire il suo numero telefonico per ricevere infomazioni aggiuntive.\newline
Per potere poi usufruire di convenzioni rilasciate solo a studenti sarà necessario il codice mi matricola in caso di universitari o l'id della carta Iostudio, e le agevolazioni verrano fornite solo per validità da mensili in su.\newline
Nel caso di turisti non è necessaria nessuna informazione aggiuntiva, ma l'agevolazione verrà fornita solo sulla validità giornaliera, weekend o settimanale.\newline
L'utenza disporrà di una tessera personale, che verrà utilizzata per identificare l'utente in tutte le operazioni che effettuerà, questa tessera avrà una validità data dal piano presente e non potrà mai rimanere senza piano. Un utente con un piano ancora valido (attivo) potrà usufruire di tutti i servizi, mentre un utente con una tessera non più valida (piano scaduto) potrà solo comperare un nuovo piano ad un prezzo prefissato, da cui verrà tolto il 15\% per i turisti che usufruiscono dell'agevolazione e del 10\% per i studenti che usufruiscono dell'agevolazione.\newline
Le biciclette, identificate da un codice personale, possono essere a pedalata assistita (elettriche) oppure con un seggiolino (la maggior parte), mentre in base al loro stato possono essere suddivise tra biciclette in uso, cioè che possono essere nolleggiate o che stanno venendo noleggiate, biciclette in deposito o biciclette danneggiate; lo stato di una bicicletta non sempre corrisponde alla propria posizione, infatti una bicicletta danneggiata può sia essere presente in magazzino sia in una colonnina.\newline
Una bicicletta in uso verrà noleggiata e depositata in una colonnina, identificata da un codice numerico e dalla stazione di cui fa parte; una colonnina presenta uno stato, che può essere occupata se vi è presente una bicicletta, libera se non vi è presente una bicicletta, oppure danneggiata se la colonnina o la bicicletta presente sono daneggiate.\newline
Una stazione é formata da più colonnine e si utilizza un numero per identificarla tra tutte le stazioni della città; una stazione avrà un proprio indirizzo per facilitare la ricerca nelle mappe, e in una specifica area della città ci possono essere anche più stazioni vicine. L'utente interaggirà con la stazione per sapere i stati delle varie colonnine presenti ma non si interesserà di quale colonnina avrà tale stato.\newline
Il noleggio di una bicicletta, richiede che venga ricordato l'orario in cui è avvenuto il prelievo della bicicletta e in quale colonnina è stata presa, e una volta terminato l'orario e la colonnina in cui è stata depositata la bicicletta; per questo il noleggio può essere diviso in noleggi in corso, che quindi identificano i prelievi delle biciclette non ancora depositate, oppure i noleggi terminati, che hanno anche il deposito.

\subsection{Operazioni tipiche}
Operazioni tipiche eseguite dall'utente :
\paragraph{Noleggio di una bicicletta} Il noleggio di una bicicletta interessa una bicicletta che ha lo stato in uso, ed sta attualmente occupando una colonnina perchè non è già in noleggio, e lascia la colonnina nello stato di libera. Questo genera un noleggio di tipo in corso, dove viene salvata la bicicletta, la colonnina di prelievo e l'orario di prelievo. Un utente può avere un singolo noleggio attivo alla volta.
\paragraph{Deposito di una bicicletta} Il deposito di una bicicletta interessa sempre una bicicletta con lo stato in uso, e modifica un noleggio di tipo in uso in uno di tipo terminato, salvando quindi la colonnina in cui viene depositata la bicicletta e l'orario di deposito; questo porta la colonnina, che prima era nello stato libera, nello stato occupata.
\paragraph{Controllo dello stato di una stazione o area} Il controllo dello stato implica un controllo di ogni singola colollonina, e ritorna quanti spazi, liberi e quante biciclette e di quale tipo sono presenti sulla data stazione , per l'area si cerca in tutte le stazioni di quell'area.
\paragraph{Segnalazione di mancanza di biciclette o spazi vuoti} Una segnalazione di tipo mancanza viene chiamata dall'utente tramite il sito oppure il numero telefonico su una data stazione e comporterà uno o più trasporti per assolvere la segnalazione.
\paragraph{Segnalazione di rottura} Una segnalazione di rottura viene chiamata dall'utente da una singola colonnina o dal sito e comporterà un ordine di manutenzione che potrà interessare la colonnina o la bicicletta.
\par Operazioni tipiche non effettuate dall'utente :
\paragraph{Trasporto di una bicicletta} Il trasporto può avvenire perché segnalato da un utente oppure per esigenze interne, e comporta il movimento di una bicicletta in tre modi diversi: tra due colonnine di diverse stazioni, dal magazzino ad una colonnina oppure da una colonnina al magazzino. Il trasporto di una bicicletta cambia sicuramente lo stato della colonnina interessata, e forse cambia anche lo stato della bicicletta.
\paragraph{Manutenzione di una colonnina o di una bicicletta} La manutenzione può avviene a seguito di una segnalazione oppure può essere richiesta dall'azienda e prevede un controllo della bicicletta o della colonnina, che verrà riassunto in una descrizione dei danni presenti che sono stati riparati, e dall'orario di fine manutenzione. Si nota che la manutenzione di una bicicletta interessa la colonninna su cui è stata presente, e che per semplicità ed efficienza viene sempre controllata sia la colonnina che la bicicletta se presente.Un'operazione di matunenzione comporterà per la colonnina e la bicicletta un cambio di stato (erano daneggiate, ora non lo sono più).
\par Operazioni dell'area amministrativa :
\paragraph{resoconto dei lavori di manutenzione} E' visibile dall'area amministrativa un resoconto dei lavori di manutenzione alle biciclette e alle colonnine dello scorso mese. Con riportati le date e gli orari.
\paragraph{storico di una bicicletta} Può venire richiesto dall'area amministrativa le manutenzioni di una bicicletta, come gli utenti che l'hanno noleggiata e il tempo in cui è stata noleggiata.
\paragraph{lista dei mezzi danneggiati} Può venire richiesta una lista delle biciclette danneggiate, con la stazione e l'area in cui sono presenti.

\subsection{Nuovi requisiti}
Dopo un'esposizione e un ragionamento sulle operazioni da eseguire, si possono delineare delle prime modifiche da effettuare alla descrizione testuale :
\paragraph{Operazioni di diverso spessore} Si nota che operazioni quali noleggio-deposito, segnalazione, trasporto e manutentione non sono semplici operazioni unitarie, ma coinvolgono molteplici operazioni.
\paragraph{modifica dello stato della bicicletta} Lo stato di una bicicletta non può venire rappresentato solo con in uso, in magazzino o danneggiata; Si decide di utilizzare 2 caratteristiche : lo stato della bicicletta (operativa o danneggiata) e la sua posizione (magazzino o in uso); Questo comporta un miglior rappresentazione dello stato della bicicletta.
\paragraph{Realtà su due livelli} Si nota che si potrebbe dividere la realtà d rappresentare in due parti: una parte si occupa dello stato attuale del sistema, quindi di dove attualmente sono le bici, della presenza di collonnine vuote o piene, etc mentre un'altra si occupa dello storico di queste informazioni, quindi dei noleggi terminati, dei trasorti effettuati, delle manutenzioni, etc.

\section{Progettazione concettuale}
%
%top-down , bottom-up, inside-out , misto
% credo di stare usando un inside-out modellando a mano a mano quello che leggo
%
% cambiato tutto, inserisco le immagini passo a passo per ogni cosa che si fa
%
Partendo dall'inizio decido di modellare l'entità Utente.\newline
Entità : \textbf{Utente}\newline
Attributi di Utente :
\begin{itemize}
 \item nome
 \item cognome
 \item data di nascita
 \item luogo di nascita
 \item indirizzo di residenza(via e numero)
 \item città di residenza
 \item cap
 \item email (può essere nullo)
 \item numero telefonico (può essere nullo)
 \item tessera
\end{itemize}
%immagine ConUtente01
Sottoinsieme di Utente : \textbf{Studente}\newline
Attributi di Studente :
\begin{itemize}
 \item matricola
\end{itemize}
%immagine ConUtente02
Continuando a cercare informazioni sugli utenti trovo un'altro sottoinsieme.\newline
Sottoinsieme di Utente : \textbf{Turista}\newline
Rivaluto quindi Studente e Turista come delle generalizzazioni parziali esclusive di Utente.\newline
%immagine ConUtente03
Cercando le informazioni relative alla tessere dell'utente mi accorgo potrebbe essere rappresentata come una entità.\newline
Entità : \textbf{Tessera}\newline
Attributi di Tessera :
\begin{itemize}
 \item codice tesseramento (chiave)
 \item validità
 \item tipo di abbonamento o piano
\end{itemize}
%immagine ConTessera01
Riflettendo su Utente e Tessera decido di collegarli con una relazione.\newline
Relazione : \textbf{Tesseramento} tra Tessera e Utente :
\begin{itemize}
 \item da Tessera a Tesseramento 1 : 1
 \item da Utente a Tesseramento 1 : 1
\end{itemize}
%immagine ConTesseramento01
Con una veloce letta dei requisiti mi accorgo che la maggior parte delle volte che ci si riferisce ad Utente in realtà ci si riferisce alla Tessera dell'Utente, quindi prendo nota che Utente potrebbe venire identificato dalla tessera che possiede, principalmente perchè la relazione Tesseramento che li unisce ha cardinalità 1 : 1.\newline
%immagine ConUtente04
I tipi di abbonamento possibili sono gli stessi per tutte le tessere quindi il concetto di piano della tessera può venire modellato come un'entità.\newline
Entità : \textbf{Piano}\newline
Attributi di Piano :
\begin{itemize}
 \item tipologia : giornaliero, weekend, settimanale, mensile, trimestrale, semestrale o annuale
 \item costo
\end{itemize}
%immagine ConPiano01
Tessera e Piano sono ovviamente collegati da una relazione.\newline
Relazione : \textbf{Abbonamento} tra Tessera e Piano :
\begin{itemize}
 \item da Tessera a Abbonamento 1 : 1
 \item da Piano a Abbonamento 0 : N
\end{itemize}
%immagine ConAbbonamento01
Per modellare la parte principale del database, cioè il noleggio di biciclette da parte degli utenti, comincio con l'entità Bicicletta.\newline
Entità : \textbf{Bicicletta}\newline
Attributi di Bicicletta :
\begin{itemize}
 \item codice identificativo (chiave)
 \item tipo : normali, pedalata assistita o con 2 seggiolini
 \item stato : in uso(noleggiata o disponibile), in magazzino, danneggiate, in riparazione
\end{itemize}
%immagine ConBicicletta01
Lo stato di una bicicletta non viene rappresntato bene da un attributo, quindi prima di tutto suddivido le biciclette tra in servizio e ferme.\newline
Generalizzazione completa esclusiva di Bicicletta : \textbf{In servizio} e \textbf{Ferme}\newline
%immagine ConBicicletta02
Una bicicletta in servizio è sempre o noleggiata o noleggiabile :\newline
Generalizzazione completa esclusiva di In servizio : \textbf{In uso} e \textbf{in attesa}\newline
%immagine ConInservizio01
Una bicicletta è In uso lo perchè noleggiata da un utente con la sua tessera, quindi inserisco questa relazione.\newline
Relazione : \textbf{Noleggio} tra Tessera e In uso :
\begin{itemize}
 \item da Tessera a Noleggio 0 : 1
 \item da In uso a Noleggio 1 : 1
\end{itemize}
Un Noleggio però deve contenere anche la colonnina da cui è stata prelevata la bici, la data e l'orario di prelievo.\newline
Attributi di Noleggio :
\begin{itemize}
 \item colonnina
 \item data e ora
\end{itemize}
%immagine ConNoleggio01
Allo stesso modo una bicicletta è In attesa se non noleggiata da nessun Utente, cioè è ferma sulla colonnina.\newline
Relazione : \textbf{Posizione} tra Colonnina e In attesa :
%non mi piace per nulla il nome della relazione
\begin{itemize}
 \item da Colonnina a Posizione 0 : 1
 \item da In attesa a Posizione 1 : 1
\end{itemize}
%immagine ConPosizione01
Mi accorgo che l'attributo colonnina di Noleggio in realtà è una relazione con Colonnina, il che fa diventare Noleggio una relazione tra tre entità.\newline
Relazione : \textbf{Noleggio} tra Tessera, In uso e Colonnina :
\begin{itemize}
 \item da Tessera a Noleggio 0 : 1
 \item da In uso a Noleggio 1 : 1
 \item da Colonnina a Noleggio 1 : 1
\end{itemize}
%immagine ConNoleggio02
Anche per le biciclette Ferme posso trovare una suddivisione.\newline
Generalizzazione completa esclusiva di Ferme : \textbf{Magazzino} e 
%%%%%%%%%%%%%%%%%%%%%%%%%%%%%%%%%%%%%%%%%%%%%%%%%%%%%%%%%%fino a qui

Entità : \textbf{Stazione}\newline
Attributi di \textbf{Stazione} :
\begin{itemize}
 \item codice (chiave)
 \item area (ex: Stazione, Prato... dove sono presenti più stazioni)
 \item indirizzo (via e numero)
 \item postazioni presenti (numero di colonnine)
 \item stato attuale delle colonnine (occupate, libere, tipologia di bici presenti)
 \end{itemize}
Relazione : \textbf{Appartenenza} tra \textbf{Stazione} e \textbf{Colonnina}
\begin{itemize}
 \item da \textbf{Stazione} a \textbf{Colonnina} 1 : N
 \item da \textbf{Colonnina} a \textbf{Stazione} 1 : 1
\end{itemize}
Mi accorgo che il concetto 'stato attuale delle colonnine' non riesco a rappresentarlo, quindi decido di sviluppare questo attributo in una nuova Entità e Relazione.\newline
Entità : \textbf{Colonnina}\newline
Attributi di \textbf{Colonnina} :
\begin{itemize}
 \item codice numerico
\end{itemize}
Generalizzazione di \textbf{Colonnina} : \textbf{Colonnina occupata}\newline
Attributi di \textbf{Colonnina occupata} :
\begin{itemize}
 \item tipologia bicicletta
\end{itemize}
Relazione : \textbf{Posizionamento} tra \textbf{Colonnina occupata} e \textbf{Bicicletta}
\begin{itemize}
 \item da \textbf{Colonnina occupata} a \textbf{Bicicletta} 1 : 1
 \item da \textbf{Bicicletta} a \textbf{Colonnina occupata} 1 : 1
\end{itemize}
Per rappresentare il concetto della manutenzione delle biciclette decido di creare una nuova Relazione.\newline
Relazione : \textbf{Segnalazione bicicletta} tra \textbf{Utente} e \textbf{Bicicletta}
\begin{itemize}
 \item da \textbf{Utente} a \textbf{Bicicletta} 0 : N
 \item da \textbf{Bicicletta} a \textbf{Utente} 0 : N
\end{itemize}
Attributi di \textbf{Segnalazione bicicletta} :
\begin{itemize}
 \item la colonnina dove è presente la bicicletta
 \item descrizione del danno
\end{itemize}
Cercando di rappresentare il concetto della manutenzione delle colonnine mi accorgo che la Relazione \textbf{Segnalazione bicicletta} è molto simile e potrebbero essere 2 Generalizzazioni di un concetto di Segnalazione più generale da parte dell'Entità \textbf{Utente};\newline
Quindi per concettualizzare meglio il tutto decido di introdurre una nuova Entità, due Generalizzazioni e varie Relazioni.\newline
Entità : \textbf{Segnalazione}\newline
Attributi di \textbf{Segnalazione} :
\begin{itemize}
 \item descrizione del danno presente
 \item data di segnalazione
\end{itemize}
Generalizzazione di \textbf{Segnalazione} : \textbf{Segnalazione bicicletta}\newline
Generalizzazione di \textbf{Segnalazione} : \textbf{Segnalazione colonnina}\newline
%Generalizzazione completa, ma non so come scriverlo
Relazione : \textbf{Segnalamento} tra \textbf{Utente} e \textbf{Segnalazione}
\begin{itemize}
 \item da \textbf{Utente} a \textbf{Segnalazione} 0 : N
 \item da \textbf{Segnalazione} a \textbf{Utente} 1 : 1
\end{itemize}
Relazione : \textbf{Riferisce} tra \textbf{Colonnina} e \textbf{Segnalazione}
\begin{itemize}
 \item da \textbf{Colonnina} a \textbf{Segnalazione} 0 : N
 \item da \textbf{Segnalazione} a \textbf{Colonnina} 1 : 1
\end{itemize}
Relazione : \textbf{Rottura} tra \textbf{Bicicletta} e \textbf{Segnalazione}
\begin{itemize}
 \item da \textbf{Bicicletta} a \textbf{Segnalazione} 0 : N
 \item da \textbf{Segnalazione} a \textbf{Bicicletta} 1 : 1
\end{itemize}
Per tenere traccia delle \textbf{Segnalazioni} ancora attive e quelle processate si introduce una nuova Generalizzazione.\newline
Generalizzazione di \textbf{Segnalazione} : \textbf{Segnalazione pendente}\newline
Generalizzazione di \textbf{Segnalazione} : \textbf{Segnalazione risolta}\newline
%Generalizzazione completa, ma non so come scriverlo
Perché l'utente possa eseguire l'operazione di noleggio o restituzione di una bicicletta c'è bisogno di modellare una nuova Relazione.\newline
Relazione : \textbf{Noleggio} tra \textbf{Utente} e \textbf{Bicicletta}
\begin{itemize}
 \item da \textbf{Utente} a \textbf{Segnalazione} 0 : N
 \item da \textbf{Segnalazione} a \textbf{Utente} 1 : 1
\end{itemize}




%
%%%da finire
%
Legami non rappresentati :\newline
Alcuni utenti possono avere agevolazioni in base alla loro tipologia e piano che scelgono sulla loro tessera :
\begin{itemize}
 \item  i turisti per i soli piani giornaliero o weekend (-15\% grazie ad un contributo della regione)
 \item i studenti per gli abbonamenti da settimanale in su (-10\%).
\end{itemize}

\section{Progettazione logica}

\section{Implementazione dello schema logico}

\section{Query, procedure, trigger e funzioni}

\section{Interfaccia web}

\end{document}